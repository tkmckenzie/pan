\documentclass{article}
\usepackage{fullpage}
\usepackage{amsmath}
\usepackage{amssymb}
\usepackage{amsthm}
\usepackage{graphicx}
\usepackage{natbib}
\usepackage[capitalise,noabbrev]{cleveref}
\usepackage{enumitem}
\usepackage{multirow}
\usepackage{breqn}

\usepackage{tikz}
\usetikzlibrary{angles,quotes}

\def\radius{3cm}
\def\distance{5cm}
\def\appxangle{33.55731}
\def\appxheight{1.658312cm}

\def\phiangle{22.5}
\def\phiheight{0.8284271cm}

\begin{document}
	
First, let $r$ denote the radius of both circles and $d$ denote the distance between the center of the two circles. Also let $A_C$ be the total area of the circle (so $A_C = \pi r^2$) and let $A_I$ be the area of the intersection of the two circles. Given $r$, we want to find $d$ such that the area of the circle excluding the intersecting area (given by $A_C - A_I$) is equal to the intersection area $A_I$. That is,
\begin{equation}
	A_C - A_I = A_I \Rightarrow A_C = 2 A_I.
\end{equation}

We can use integration in polar coordinates to find $A_I$. First, we need to find the limits of integration by finding $\theta$ in the figure below.
\begin{center}
\begin{tikzpicture}
\coordinate (O1) at (-\distance/2,0);
\coordinate (O2) at (\distance/2,0);
\coordinate (P1) at (0, \appxheight);

\draw (O1) node[circle,inner sep=1.5pt,fill] {} circle [radius=\radius];
\draw (O2) node[circle,inner sep=1.5pt,fill] {} circle [radius=\radius];

\draw
	(O2) coordinate (xcoord) --
	node[midway,below] {$d$} (O1) --
	node[midway,above] {$r$} (P1)
	pic [draw,->,angle radius=1cm,"$\theta$"] {angle = O2--O1--P1};
\end{tikzpicture}
\end{center}
Using the definition of the cosine, we can see that $\cos(\theta) = (d / 2) / r$, which implies that $\theta = \arccos(d / (2r))$.

Next, we can find the distance between the left circle's center and the edge of the right circle, which will be denoted with $f(\varphi)$ for angle $\varphi$. This is shown in the picture below.
\begin{center}
	\begin{tikzpicture}
	\coordinate (O1) at (-\distance/2,0);
	\coordinate (O2) at (\distance/2,0);
	\coordinate (P1) at (0, \appxheight);
	\coordinate (P2) at (-0.5, \phiheight);
	
	\fill (O1) circle [radius=1.5pt];
	\draw (O2) node[circle,inner sep=1.5pt,fill] {} circle [radius=\radius];
	
	\draw
	(O2) coordinate (xcoord) --
	node[midway,below] {$d$} (O1) --
	node[midway,above] {$f(\varphi)$} (P2)
	pic [draw,->,angle radius=1.5cm,"$\varphi$"] {angle = O2--O1--P2};
	\end{tikzpicture}
\end{center}

\end{document}
